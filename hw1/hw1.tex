%
%     hw1.tex
%     Benjamin Brown (bpbrown@colorado.edu)
%     Jan 24, 2017
%
%     Problemset 1 for ASTR 5550, Obs & Stats taught at
%     University of Colorado, Boulder, Spring 2017.
%

\documentclass[12pt, preprint]{aastex6}
% formatting based on 2014 NASA ATP proposal with Jeff Oishi

%%%%%%begin preamble
\usepackage[hmargin=1in, vmargin=1in]{geometry} % Margins
\usepackage{hyperref}
\usepackage{url}
\usepackage{times}
\usepackage{natbib}
\usepackage{graphicx}
\usepackage{amsmath}
\usepackage{amsfonts}
\usepackage{amssymb}
\usepackage{pdfpages}
\usepackage{import}
\usepackage{listings}

\hypersetup{
     colorlinks   = true,
     citecolor     = gray,
     urlcolor       = blue
}

%% headers
\usepackage{fancyhdr}
\pagestyle{fancy}
\lhead{ASTR 5550}
\chead{}

\cfoot{\thepage}
\rfoot{Spring 2017}
% no hline under header
\renewcommand{\headrulewidth}{0pt}

\newcommand{\sol}{\ensuremath{\odot}}
\newcommand{\dedalus}{\href{http://dedalus-project.org}{Dedalus}}
\renewcommand{\vec}{\boldsymbol}
\newcommand{\del}{\vec{\nabla}}
% make lists compact
\usepackage{enumitem}
%\setlist{nosep}

\newenvironment{solution}{\color{blue} \textbf{Solution:} }{}

%%%%%%end preamble

\lfoot{Problem Set 1}

\begin{document}

\section*{Problem Set 1 (due Wed Feb 1).}
\begin{description}
\item[Detrending Kepler data (20 pts)] ~\\
    The \emph{Kepler} spacecraft monitored stars for several years, 
    providing amazing photometric lightcurves with precisions in
    the parts-per-million range.  This exquisitely low error is
    requisite for \emph{Kepler's} main science objective: detecting
    Earth-like planets around other stars.  The raw data that comes
    off the spacecraft however has large, long-timescale variations
    due to a variety of factors including thermal
    expansion/contraction of the telescope's optical path and 
    stars drifting out of their designated ``pixel aperture''.  These
    long-timescale variations can be removed by detrending the data in
    various ways.  Here, we're going to unleash the power of linear
    regression to remove a low-order polynomial from real
    \emph{Kepler} data.  Our target is \textbf{KIC-7200111}, 
    a Sun-like star, and we'll look at it's $3^{\textrm{rd}}$ quarter
    data\footnote{from MAST, at 
      \url{https://archive.stsci.edu/kepler/data_search/search.php}}.
    Kepler data, as with many other forms of
    astronomical data, is in FITS format.  To get started, install 
    \verb+pyfits+ by doing \verb+$pip3 install pyfits+ at your command
    line.   This problem
    will also provide your introduction to python3\footnote{I sugest you do
      all work in this class in python version 3, not version 2.},
    to \href{https://bitbucket.org/}{bitbucket},
    to plotting, and to Latex\footnote{If you need to brush up
      on your LaTeX, start with ``The not so short introduction to
      LaTeX,'' at \href{http://www.ctan.org/tex-archive/info/lshort/english/}{http://www.ctan.org/tex-archive/info/lshort/english}.
      This is an excellent resource.}.

    \emph{NOTE: the code snippet
       produces two lightcurves: an uncorrected one and a PDC
       lightcurve, which is the Kepler ``first pass" at error
       correction.  Throughout this homework, do the fitting, etc. to
       the uncorrected lightcurve.  We will compare our results to PDC at the end.}

 \begin{enumerate}
   \setcounter{enumi}{-1}
   \item Go to the MAST website, enter \textbf{7200111} in the field
     "Kepler ID" and click on the quarter~3 dataset.  At the top of
     the page, download the light curve (.fits) file. 

   \item Using the code provided, read in the Kepler FITS file data.
     Plot the lightcurve and make sure it agrees broadly 
     with the image you saw on MAST.  

    \item Using linear regression and least squares, produce
      polynomial fits for the uncorrected lightcurve.  Sample a
      variety of polynomial degrees (at least from 1 to 4).  Plot the
      resulting polynomial fits over the data.  Comment.

     \item Detrend the data by removing a $N=3$ polynomial.  Plot
       this detrended data, and on the same plot, show the PDC lightcurve from the
       \emph{Kepler} pipeline\footnote{you'll need to
         remove the mean from PDC, since detrending removes the mean
         from the raw lightcurve}.  
       Comment on the agreement between our
       crude polynomial fitting technique and PDC.
       \label{detrended data}

       \item What could be the source of the remaining wiggles in the
         stellar lightcurve?  Justify.

\end{enumerate}
\newpage
  \item[Analyzing Kepler data (20 pts)]~\\
    Here we continue on with our \emph{Kepler} explorations.
    Our target remains \textbf{KIC-7200111}.  Now it's time to try
    Fourier transforms of the data to obtain periods.

\begin{enumerate}
  \setcounter{enumi}{-1}
   \item Start with your detrended solution from Part 1: subpart \ref{detrended data}.

   \item Library FFTs (Fast Fourier Transforms) require data on 
     uniformly sampled locations.
     Use \verb+np.interp()+ to interpolate your data onto a uniform
     grid with 8192 equally spaced time points.  The function
     \verb+np.linspace()+ may be helpful.  Confirm that your
     interpolated data agrees reasonably with your original detrended
     data (plotting and chi-by-eye is sufficient here).

   \item Perform a FFT on your data, using \verb+np.fft.fft()+. 
     How do the returned coefficients in the FFT array 
     map onto temporal frequencies? (e.g., what frequency is stored in
     FFT[0]?  What frequencies are stored in increasing locations
     within the array?  What happens after we pass the mid-point in
     the array?) \emph{hint: read the numpy docs}.

   \item Form the power spectrum by multiplying the FFT by it's
     complex conjugate.  Plot the power spectrum versus frequency for
     the positive frequencies.
     Use a loglog scale.

   \item Identify the two highest peaks of power.  What frequencies
     do these correspond to?  What periods do these correspond to
     ($P=1/f$)?  What might these periods represent?

   \item The FFT assumes that your data is periodic.  Is this data
     periodic?  What problems could arise from non-periodicity?
\end{enumerate}

\item[Asteroseismology (10 pts)] ~\\
    Let's try playing with Short Cadence \emph{Kepler} data.  Go to
    the MAST webpage and search for \emph{16 Cyg a}, one of the brightest stars
    observed with \emph{Kepler}.  Find the Q7 Short Cadence (target
    type: SC) data and download the FITS file.  The target is
    \textbf{KIC-100002741}, and the file is \verb+kplr100002741-2010296114515_slc.fits+.
    Now it's time to try Fourier transforms of the data to obtain
    oscillations.
\begin{enumerate}
  \item  Use least squares to fit and remove a
     $N=3$ polynomial from the data.  Interpolate the data onto a
     uniform grid with $2^{16}=65536$ points in time.  Take the FFT
     and form the powerspectrum.  Plot the powerspectrum on a log-log
     scale.  Comment.  

     \item Plot the powerspectrum a second time on a
     linear-linear scale, with units of seconds$^{-1}$ on the frequency
     axis (previously you used day$^{-1}$).  Zoom in on the frequency range from
     $1.5\times 10^{-3}$--$3.0\times 10^{-3}$ sec$^{-1}$, and choose a y-scale
     that emphasizes the peaks.  How does the peak oscillation
     frequency compare to the peak of the solar helioseismic
     oscillations?  

     \item \emph{Extra credit:} what could you infer about the
     qualitative stellar properties of \emph{Cyg 16 }a compared to the Sun based on the
     oscillation frequencies?

\end{enumerate}

\textbf{To hand in:} answers to the above problems, plus labelled,
clearly designed figures for all requested plots.  
We would also like your final code producing the figures.  Hand code in via
\href{bitbucket.org}{https://bitbucket.org/}\footnote{We will go over
  version control in class, but please set up an account on
  \href{bitbucket.org}{https://bitbucket.org/} now.   Use your
  colorado.edu e-mail address and request an ``academic'' license.
If you want to go further, install the ``mercurial'' DVCS software on
your computer.  You will very likely need Mac or Linux throughout this
course.}; we might consider other options for handing in code,
depending on preferences from our grader.

\end{description}

%
%     hw1_code.tex
%     Benjamin Brown (bpbrown@colorado.edu)
%     Jan 24, 2017
%
%     Problemset 1 for ASTR/ATOC 5550, Mathematical Methods, taught at
%     University of Colorado, Boulder, Spring 2017.
%
%     Python code importing block.
%


\definecolor{codegreen}{rgb}{0,0.6,0}
\definecolor{codegray}{rgb}{0.5,0.5,0.5}
\definecolor{codepurple}{rgb}{0.58,0,0.82}
\definecolor{backcolour}{rgb}{0.95,0.95,0.92}
\definecolor{codered}{rgb}{0.6,0,0}
\definecolor{codeblue}{rgb}{0,0,0.6}

\lstdefinestyle{mystyle}{
    backgroundcolor=\color{backcolour},   
    commentstyle=\color{codered},
    keywordstyle=\color{codepurple},
    numberstyle=\tiny\color{codegray},
    stringstyle=\color{codepurple},
    basicstyle=\footnotesize,
    breakatwhitespace=false,         
    breaklines=true,                 
    captionpos=b,                    
    keepspaces=true,                 
    numbers=left,                    
    numbersep=5pt,                  
    showspaces=false,                
    showstringspaces=false,
    showtabs=false,                  
    tabsize=2
}

\lstset{style=mystyle}


\lstinputlisting[language=Python]{hw1.py}


\end{document}
